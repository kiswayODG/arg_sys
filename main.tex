\documentclass{rapportECL}
\usepackage{lipsum}
\usepackage{svg}
\usepackage{algorithm}
\usepackage{algorithmic}

\title{Rapport ECL - Template} %Titre du fichier

\begin{document}

%----------- Informations du rapport ---------

\titre{Implémentation d'Algorithmes pour les Systèmes d'Argumentation Abstraits} %Titre du fichier .pdf
\UE{Représentation des Connaissances et Raisonnement} %Nom de la UE
\sujet{ Introduction à l’Argumentation - Projet} %Nom du sujet

\enseignant{Elise \textsc{Bonzon}} %Nom de l'enseignant

\eleves{Hugo \textsc{Convert} \\
		Kiswendsida \textsc{OUEDRAOGO} } %Nom des élèves

%----------- Initialisation -------------------
        
\fairemarges %Afficher les marges
\fairepagedegarde %Créer la page de garde
\tabledematieres %Créer la table de matières

%------------ Corps du rapport ----------------
\begin{abstract}
	Ce rapport présente le travail réalisé dans le cadre du développement d’un outil informatique dédié à la résolution de problèmes 
	dans le domaine des systèmes d’argumentation abstraits. Nous avons commencé par une exploration approfondie des algorithmes de 
	calcul des extensions complètes et stables, en nous appuyant sur des travaux de recherche issus de la littérature scientifique 
	sur l’argumentation. Ces algorithmes ont été décrits en détail, accompagnés d’explications sur leurs mécanismes et les structures 
	de données utilisées pour leur implémentation.

    Grâce à ces algorithmes, nous avons pu répondre aux problématiques posées dans le sujet du projet, telles que la détermination 
	des extensions d’un système d’argumentation donné et l’évaluation de l’appartenance d’un argument à une extension.

    En complément, nous avons développé une interface graphique conviviale permettant de visualiser les systèmes d’argumentation, 
	leurs arguments, et les relations entre eux, offrant ainsi une meilleure interaction utilisateur et facilitant l’analyse des 
	résultats. Ce rapport détaille les méthodologies employées, les choix techniques effectués et les résultats obtenus, illustrant 
	ainsi notre contribution à la mise en œuvre pratique des concepts théoriques de l’argumentation.
\end{abstract}

\newpage % Crée une nouvelle page après l'abstract


\section{Introduction} 
Pour la réalisation de ce projet, une revue de littérature s’est naturellement imposée. Parmi les ressources pertinentes, nous avons trouvé plusieurs travaux particulièrement intéressants. 

En premier lieu, nous avons étudié le document sur l'argumentation computationnelle, notamment le cours d'Élise Bonzon, qui nous a servi de base pour comprendre les concepts fondamentaux. 
Nous avons également exploré l'article \textit{An Introduction to Argumentation Semantics}, qui propose des définitions assez approfondies des différentes sémantiques de l'argumentation. 
Enfin, l’étude des algorithmes pour les sémantiques d'argumentation, présentée dans \textit{Algorithms for Argumentation Semantics: Labeling Attacks as a Generalization of Labeling Arguments}, 
nous a permis d’approfondir nos connaissances sur les algorithmes utilisés dans ce domaine et ainsi implémenter la solution du projet présent. 

Cette exploration nous a permis de définir les objectifs du projet, présentés dans la section suivante.

\section{Objectifs et fonctionnalités attendues du projet}
Ce projet vise à développer un outil capable de résoudre des problèmes spécifiques liés aux systèmes d'argumentation abstraits (AF). 
Un système d'argumentation est défini par un ensemble d'arguments (A) et une relation d'attaque (R) entre ces arguments. 
L'objectif est de fournir une solution capable d’identifier les extensions complètes et stables, et évaluer l’appartenance 
d’un argument à ces extensions selon des critères crédibles ou sceptiques.

Le programme lit un système d'argumentation à partir d'un fichier texte formaté, analyse les données, et produit des résultats conformes selon la sémantique utilisée. 


En complément, le projet propose une interface visuelle conviviale permettant de visualiser les arguments, leurs relations à travers des graphes et les résultats des analyses, facilitant ainsi l’interaction utilisateur et l’interprétation des résultats.

\section{Algorithmes et structures de données} 


\subsection{Algorithmes}
Les algorithmes utilisés pour l'implémentation de ce projet sont spécifiquement ceux dédiés à l'énumération des extensions complètes 
et stables. Ces algorithmes sont issus des travaux présentés dans l'article 
\textit{"Algorithms for Argumentation Semantics: Labeling Attacks as a Generalization of Labeling Arguments"} [1].

\begin{itemize}
    \item Énumération des extensions stables.
    
	L'algorithme ci dessous de identifie toutes les extensions stables d’un AF (A,R) correspond[1] (p.642).
	\begin{algorithm}
		\caption{Enumerating all stable extensions of an AF $(A, R)$}
		\begin{algorithmic}[1]
			\STATE $Lab : A \to \{IN, OUT, MUST OUT, BLANK\}$; $Lab \gets \emptyset$
			\FORALL{$x \in A$}
				\STATE $Lab \gets Lab \cup \{(x, BLANK)\}$
			\ENDFOR
			\STATE $Estable \subseteq 2^A$; $Estable \gets \emptyset$
			\STATE \textbf{call} find-stable-extensions($Lab$)
			\STATE \textbf{report} $Estable$ is the set of all stable extensions
			\STATE \textbf{procedure} find-stable-extensions($Lab$)
			\WHILE{there exists $y \in A$ such that $Lab(y) = BLANK$}
				\IF{there exists $y$ such that $Lab(y) = BLANK$ and $\forall z \in \{y\}^- : Lab(z) \in \{OUT, MUST OUT\}$}
					\STATE select $y$ such that $Lab(y) = BLANK$
				\ELSE
					\STATE select $y$ such that $Lab(y) = BLANK$ and $\forall z : Lab(z) = BLANK$, 
					\STATE $\left| \{x : x \in {y}^+ \land Lab(x) \neq OUT \} \right| \geq \left| \{x : x \in {z}^+ \land Lab(x) \neq OUT \} \right|$
				\ENDIF
				\STATE $Lab' \gets Lab$
				\STATE $Lab'(y) \gets IN$
				\FORALL{$z \in {y}^+$}
					\STATE $Lab'(z) \gets OUT$
				\ENDFOR
				\FORALL{$z \in {y}^-$}
					\IF{$Lab'(z) = BLANK$}
						\STATE $Lab'(z) \gets MUST OUT$
					\ENDIF
					\IF{for all $w \in {z}^-$, $Lab'(w) \neq BLANK$}
						\STATE $Lab(y) \gets MUST OUT$
					\ENDIF
				\ENDFOR
				\STATE \textbf{goto} line 7
				\STATE \textbf{call} find-stable-extensions($Lab'$)
				\IF{there exists $z \in {y}^-$ such that $Lab(z) = BLANK$}
					\STATE $Lab(y) \gets MUST OUT$
				\ELSE
					\STATE $Lab \gets Lab'$
				\ENDIF
				\IF{for all $x$, $Lab(x) \neq MUST OUT$}
					\STATE $S \gets \{x : Lab(x) = IN\}$
					\STATE $Estable \gets Estable \cup \{S\}$
				\ENDIF
			\ENDWHILE
		\end{algorithmic}
	\end{algorithm}
	\begin{quote}
		La notation \( y^- \) désigne l'ensemble des attaquants de \( y \), c'est-à-dire les éléments qui attaquent \( y \) dans l'argumentation. La notation \( y^+ \), quant à elle, représente l'ensemble des attaqués de \( y \), c'est-à-dire les éléments que \( y \) attaque.
	\end{quote}	
\end{itemize}


\lipsum[3-4] %Effacer cette ligne et écrire le texte souhaité

\section{Deuxième section}

\lipsum[3-5] %Effacer cette ligne et écrire le texte souhaité

%------------- Commandes utiles ----------------

\section{Quelques commandes}

Voici quelques commandes utiles :

%------ Pour insérer et citer une image centralisée -----

%\insererfigure{logos/logo.svg}{3cm}{Légende de la figure}{Label de la figure}
% Le premier argument est le chemin pour la photo
% Le deuxième est la hauteur de la photo
% Le troisième la légende
% Le quatrième le label
Ici, je cite l'image \ref{fig: Label de la figure}


%------- Pour insérer et citer une équation --------------

\begin{equation} \label{eq: exemple}
\rho + \Delta = 42
\end{equation}

L'équation \ref{eq: exemple} est cité ici. 

% ------- Pour écrire des variables ----------------------

Pour écrire des variables dans le texte, il suffit de mettre le symbole \$ entre le texte souhaité comme : constante $\rho$. 


\end{document}
