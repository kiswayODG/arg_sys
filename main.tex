\documentclass{rapportECL}
\usepackage{lipsum}
\usepackage{svg}
\title{Rapport ECL - Template} %Titre du fichier

\begin{document}

%----------- Informations du rapport ---------

\titre{Implémentation d'Algorithmes pour les Systèmes d'Argumentation Abstraits} %Titre du fichier .pdf
\UE{Représentation des Connaissances et Raisonnement} %Nom de la UE
\sujet{ Introduction à l’Argumentation - Projet} %Nom du sujet

\enseignant{Elise \textsc{Bonzon}} %Nom de l'enseignant

\eleves{Hugo \textsc{Convert} \\
		Kiswendsida \textsc{OUEDRAOGO} } %Nom des élèves

%----------- Initialisation -------------------
        
\fairemarges %Afficher les marges
\fairepagedegarde %Créer la page de garde
\tabledematieres %Créer la table de matières

%------------ Corps du rapport ----------------
\begin{abstract}
	Ce rapport présente le travail réalisé dans le cadre du développement d’un outil informatique dédié à la résolution de problèmes 
	dans le domaine des systèmes d’argumentation abstraits. Nous avons commencé par une exploration approfondie des algorithmes de 
	calcul des extensions complètes et stables, en nous appuyant sur des travaux de recherche issus de la littérature scientifique 
	sur l’argumentation. Ces algorithmes ont été décrits en détail, accompagnés d’explications sur leurs mécanismes et les structures 
	de données utilisées pour leur implémentation.

    Grâce à ces algorithmes, nous avons pu répondre aux problématiques posées dans le sujet du projet, telles que la détermination 
	des extensions d’un système d’argumentation donné et l’évaluation de l’appartenance d’un argument à une extension.

    En complément, nous avons développé une interface graphique conviviale permettant de visualiser les systèmes d’argumentation, 
	leurs arguments, et les relations entre eux, offrant ainsi une meilleure interaction utilisateur et facilitant l’analyse des 
	résultats. Ce rapport détaille les méthodologies employées, les choix techniques effectués et les résultats obtenus, illustrant 
	ainsi notre contribution à la mise en œuvre pratique des concepts théoriques de l’argumentation.
\end{abstract}

\newpage % Crée une nouvelle page après l'abstract


\section{Première section} 

\lipsum[3-4]%Effacer cette ligne et écrire le texte souhaité

\subsection{Subsection}

\lipsum[3-4] %Effacer cette ligne et écrire le texte souhaité

\section{Deuxième section}

\lipsum[3-5] %Effacer cette ligne et écrire le texte souhaité

%------------- Commandes utiles ----------------

\section{Quelques commandes}

Voici quelques commandes utiles :

%------ Pour insérer et citer une image centralisée -----

%\insererfigure{logos/logo.svg}{3cm}{Légende de la figure}{Label de la figure}
% Le premier argument est le chemin pour la photo
% Le deuxième est la hauteur de la photo
% Le troisième la légende
% Le quatrième le label
Ici, je cite l'image \ref{fig: Label de la figure}


%------- Pour insérer et citer une équation --------------

\begin{equation} \label{eq: exemple}
\rho + \Delta = 42
\end{equation}

L'équation \ref{eq: exemple} est cité ici. 

% ------- Pour écrire des variables ----------------------

Pour écrire des variables dans le texte, il suffit de mettre le symbole \$ entre le texte souhaité comme : constante $\rho$. 


\end{document}
